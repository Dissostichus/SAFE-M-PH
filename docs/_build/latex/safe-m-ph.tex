%% Generated by Sphinx.
\def\sphinxdocclass{report}
\documentclass[letterpaper,10pt,french]{sphinxmanual}
\ifdefined\pdfpxdimen
   \let\sphinxpxdimen\pdfpxdimen\else\newdimen\sphinxpxdimen
\fi \sphinxpxdimen=.75bp\relax
\ifdefined\pdfimageresolution
    \pdfimageresolution= \numexpr \dimexpr1in\relax/\sphinxpxdimen\relax
\fi
%% let collapsible pdf bookmarks panel have high depth per default
\PassOptionsToPackage{bookmarksdepth=5}{hyperref}

\PassOptionsToPackage{booktabs}{sphinx}
\PassOptionsToPackage{colorrows}{sphinx}

\PassOptionsToPackage{warn}{textcomp}
\usepackage[utf8]{inputenc}
\ifdefined\DeclareUnicodeCharacter
% support both utf8 and utf8x syntaxes
  \ifdefined\DeclareUnicodeCharacterAsOptional
    \def\sphinxDUC#1{\DeclareUnicodeCharacter{"#1}}
  \else
    \let\sphinxDUC\DeclareUnicodeCharacter
  \fi
  \sphinxDUC{00A0}{\nobreakspace}
  \sphinxDUC{2500}{\sphinxunichar{2500}}
  \sphinxDUC{2502}{\sphinxunichar{2502}}
  \sphinxDUC{2514}{\sphinxunichar{2514}}
  \sphinxDUC{251C}{\sphinxunichar{251C}}
  \sphinxDUC{2572}{\textbackslash}
\fi
\usepackage{cmap}
\usepackage[T1]{fontenc}
\usepackage{amsmath,amssymb,amstext}
\usepackage{babel}



\usepackage{tgtermes}
\usepackage{tgheros}
\renewcommand{\ttdefault}{txtt}



\usepackage[Sonny]{fncychap}
\ChNameVar{\Large\normalfont\sffamily}
\ChTitleVar{\Large\normalfont\sffamily}
\usepackage{sphinx}

\fvset{fontsize=auto}
\usepackage{geometry}


% Include hyperref last.
\usepackage{hyperref}
% Fix anchor placement for figures with captions.
\usepackage{hypcap}% it must be loaded after hyperref.
% Set up styles of URL: it should be placed after hyperref.
\urlstyle{same}

\addto\captionsfrench{\renewcommand{\contentsname}{Contents:}}

\usepackage{sphinxmessages}
\setcounter{tocdepth}{1}



\title{SAFE\sphinxhyphen{}M\sphinxhyphen{}PH}
\date{juin 30, 2025}
\release{1.0}
\author{SAFE\sphinxhyphen{}M}
\newcommand{\sphinxlogo}{\vbox{}}
\renewcommand{\releasename}{Version}
\makeindex
\begin{document}

\ifdefined\shorthandoff
  \ifnum\catcode`\=\string=\active\shorthandoff{=}\fi
  \ifnum\catcode`\"=\active\shorthandoff{"}\fi
\fi

\pagestyle{empty}
\sphinxmaketitle
\pagestyle{plain}
\sphinxtableofcontents
\pagestyle{normal}
\phantomsection\label{\detokenize{index::doc}}


\noindent\sphinxincludegraphics{{fig_pH_metre}.png}
\begin{itemize}
\item {} 
\sphinxAtStartPar
Ce programme permet de contrôler un ph\sphinxhyphen{}mètre arduino équipé d’une sonde de température PT100.

\item {} 
\sphinxAtStartPar
Il résulte d’un travail collectif effectué par des étudiants de Licence 3 de l’Institut de physique du globe de Paris.

\item {} 
\sphinxAtStartPar
Il est distribué sous la licence créative common CC\sphinxhyphen{}by\sphinxhyphen{}SA 4.0

\item {} \begin{description}
\sphinxlineitem{Pour le citer:}
\sphinxAtStartPar
Chardon, T., Gauthier\sphinxhyphen{}Brouard, T., Lu, C., Palmieri, C., de Singly, V., Lumembe, O., Métivier, F.,  Baugas\sphinxhyphen{}Villers, O., Bijon, V., Charles\sphinxhyphen{}Nicolas, A., Chin, C., Fossaert, H., Hallé, S., Henry\sphinxhyphen{}Gonzalez, M., Le Liorzou, C., Leroy, L., Marchaland Le Bihan, S., Monti, V., Pasquet, T., Perrenx, L., Poirier, M., Sauvage, D., Sookwhan, N., \& Thommy, G. (2024). SAFE\sphinxhyphen{}M\sphinxhyphen{}PH Un pH\sphinxhyphen{}mètre low cost pour l’enseignement {[}Computer software{]}.

\end{description}

\end{itemize}

\sphinxstepscope


\chapter{Installation}
\label{\detokenize{install:installation}}\label{\detokenize{install::doc}}

\section{Emplacement du programme}
\label{\detokenize{install:emplacement-du-programme}}
\sphinxAtStartPar
La première étape consiste à installer le programme permettant d’utiliser le pH mètre.
Ce programme est en accès libre sur le site internet GitHub sur le compte \sphinxhref{https://github.com/fmetivier/pH\_meter\_V2.0}{fmetivier}.

\sphinxAtStartPar
\sphinxurl{https://github.com/fmetivier/pH\_meter\_V2.0}


\section{Charger le répertoire}
\label{\detokenize{install:charger-le-repertoire}}
\sphinxAtStartPar
Afin d’utiliser le programme vous avez besoin de charger le répertoire dans lequel il se trouve. Ce répertoire
contient l’ensemble des codes et dossiers nécessaires à la bonne utilisation des pH mètres.

\sphinxAtStartPar
Vous pouvez cloner le répertoire en saisissant directement dans le terminal la commande:

\begin{sphinxVerbatim}[commandchars=\\\{\}]
\PYG{n}{git} \PYG{n}{clone} \PYG{n}{https}\PYG{p}{:}\PYG{o}{/}\PYG{o}{/}\PYG{n}{github}\PYG{o}{.}\PYG{n}{com}\PYG{o}{/}\PYG{n}{SAFE}\PYG{o}{\PYGZhy{}}\PYG{n}{M}\PYG{o}{/}\PYG{n}{SAFE}\PYG{o}{\PYGZhy{}}\PYG{n}{M}\PYG{o}{\PYGZhy{}}\PYG{n}{PH}\PYG{o}{.}\PYG{n}{git}
\end{sphinxVerbatim}

\begin{sphinxadmonition}{note}{Note:}
\sphinxAtStartPar
Attention, avant d’importer le programme vérifiez bien que vous vous situez dans votre répertoire de travail.
\end{sphinxadmonition}


\section{Mise en place du pH mètre}
\label{\detokenize{install:mise-en-place-du-ph-metre}}
\sphinxAtStartPar
Avant de lancer le programme, branchez le pH\sphinxhyphen{}mètre avec le cable d’alimentation à un des ports USB
de l’ordinateur. Branchez ensuite la sonde pH sur le boitier du pH\sphinxhyphen{}mètre. Devissez le capuchon de la
sonde, rincez la avec de l’eau distillée ou de l’eau claire si vous n’avez pas d’eau distillée
à votre disposition. Séchez délicatement puis plongez le pH\sphinxhyphen{}mètre dans la solution à mesurer.


\section{Démarrage du programme}
\label{\detokenize{install:demarrage-du-programme}}
\sphinxAtStartPar
Vous êtes maintenant prêt à faire fonctionner le pH\sphinxhyphen{}mètre. Vous devez avoir un dossier nommé \sphinxcode{\sphinxupquote{pH\_meter\_V2.0}}, qui contient :
\begin{description}
\sphinxlineitem{\sphinxcode{\sphinxupquote{Programme pH mètre\_V2.py}}:}
\sphinxAtStartPar
Il s’agit du programme principal permettant d’intéragir avec le pH\sphinxhyphen{}mètre au travers d’un interface dans le terminal.

\sphinxlineitem{\sphinxcode{\sphinxupquote{lib\_pH.py}}:}
\sphinxAtStartPar
Ce code contient l’ensemble des fonctions nécessaire au fonctionnement du programme principal.

\sphinxlineitem{\sphinxcode{\sphinxupquote{DATA}}:}
\sphinxAtStartPar
Ces dans ce dossier que vont être enregistrées les données acquisent lors des mesures de pH.

\sphinxlineitem{\sphinxcode{\sphinxupquote{CALIB}}:}
\sphinxAtStartPar
Ce dossier contient les données de toutes les calibrations du pH mètre effectuées, ce qui permet notamment de les réutiliser.

\sphinxlineitem{\sphinxcode{\sphinxupquote{FIGURES}}:}
\sphinxAtStartPar
Comme son nom l’indique, ce dossier regroupe l’ensemble des graphique permettant de visualiser les données mesurées et de suivre l’évolution du pH
des solutions analysées.

\sphinxlineitem{\sphinxcode{\sphinxupquote{send\_ph\_T\_Uno}}:}
\sphinxAtStartPar
Ce dossier contient le script à charger sur la carte Arduino permettant de communiquer les données de voltage et de température, mesurées par les sondes, à l’ordinateur.

\sphinxlineitem{\sphinxcode{\sphinxupquote{Fritzing}}:}
\sphinxAtStartPar
Contient une image représentant l’agencement et le branchement des différents composants du pH mètre.

\sphinxlineitem{\sphinxcode{\sphinxupquote{compare.py}}:}
\sphinxAtStartPar
Ce programme permet d’évaluer la précision et fiabilité des pH\sphinxhyphen{}mètres construits à partir de carte Arduino par rapport à un pH mètre de laboratoire Hanna.

\sphinxlineitem{\sphinxcode{\sphinxupquote{\_\_pycache\_\_}}:}
\sphinxAtStartPar
Ce répertoire automatiquement créé par Python stocke les fichiers compilés des modules Python utilisés.

\sphinxlineitem{\sphinxcode{\sphinxupquote{Mauel\_d\_utilisation\_pH\_metre.pdf}}:}
\sphinxAtStartPar
Cours manuel explicitant l’utilisation du pH mètre.

\sphinxlineitem{\sphinxcode{\sphinxupquote{CITATION.cff}}:}
\sphinxAtStartPar
Métadonnées à utiliser pour citer cd logiciel si vous souhaités l’utiliser.

\sphinxlineitem{\sphinxcode{\sphinxupquote{README.md}}:}
\sphinxAtStartPar
Documentation du pH mètre.

\end{description}

\sphinxAtStartPar
Essayons de lancer le programme pour voir comment celui\sphinxhyphen{}ci fonctionne. Saisissez simplement:

\begin{sphinxVerbatim}[commandchars=\\\{\}]
\PYG{c+c1}{\PYGZsh{} A l\PYGZsq{}intérieur du dossier pH\PYGZus{}meter\PYGZus{}V2/}
\PYG{n}{python3} \PYG{l+s+s1}{\PYGZsq{}}\PYG{l+s+s1}{Programme pH mètre\PYGZus{}V2.py}\PYG{l+s+s1}{\PYGZsq{}}
\end{sphinxVerbatim}

\sphinxAtStartPar
Cela devrait lancer le programme dans le terminal et vous dire que vous que la connexion avec l’Arduino est établie,
vous donner les paramètres de régression de la calibration par défaut du pH mètre ainsi qu’une figure de la courbe de calibration. De plus le programme affiche l’interface
du MENU PRINCIPAL.

\sphinxAtStartPar
Vous devriez obtenir:

\begin{sphinxVerbatim}[commandchars=\\\{\}]
\PYGZhy{}\PYGZgt{} python3 \PYGZsq{}Programme pH mètre\PYGZus{}V2.py\PYGZsq{}
Connexion établie avec le port /dev/ttyACM0
pH\PYGZhy{}mètre connecté  au port /dev/ttyACM0
Les paramètres a et b de notre regression linéaire sont [0.01323408 2.15989141]
Pour un voltage de 600 le pH prédit est de 10.10033889796552
0.99985

    ===========================================================================
    MENU PRINCIPAL
    ===========================================================================
    Que souhaitez\PYGZhy{}vous faire ?
    1 \PYGZhy{} Calibrer
    2 \PYGZhy{} Mesurer
    3 \PYGZhy{} Représenter graphiquement
    4 \PYGZhy{} Quitter
    ===========================================================================
    ?
\end{sphinxVerbatim}

\begin{sphinxadmonition}{warning}{Avertissement:}
\sphinxAtStartPar
Si le pH mètre n’est pas correctement branché à l’ordinateur
ou que l’Arduino ne parvient à se connecter au port d’accès un message d’erreur devrait apparaître de la forme:

\begin{sphinxVerbatim}[commandchars=\\\{\}]
\PYGZhy{}\PYGZgt{} python3 \PYGZsq{}Programme pH mètre\PYGZus{}V2.py\PYGZsq{}
/!\PYGZbs{} Port de connexion non détecté. Merci de rétablir la connexion non établie : connexion au processeur dans les réglages avant utilisation.
Attention aucun arduino disponible
\end{sphinxVerbatim}
\end{sphinxadmonition}


\section{Ajout d’un nouveau pH\sphinxhyphen{}mètre au programme}
\label{\detokenize{install:ajout-d-un-nouveau-ph-metre-au-programme}}
\sphinxAtStartPar
Si le programme ne parvient pas à se connecter avec un nouveau pH\sphinxhyphen{}mètre cela peut être du au fait qu’il n’est pas encore enregistré dans la liste des connexions
disponibles. Pour y remédier ouvrez le logiciel Arduino présent sur votre ordinateur. Une fois la fenêtre ouverte assurez vous que le pH\sphinxhyphen{}mètre est bien branché à un port USB de
votre oridnateur. En haut à gauche de la fenêtre, allez dans l’onglet \sphinxcode{\sphinxupquote{Outils}} et selectionnez l’option \sphinxcode{\sphinxupquote{Récupérer les informations de la carte}}.

\noindent\sphinxincludegraphics{{fig_arduino}.png}

\sphinxAtStartPar
En cliquant sur cette option une seconde fenêtre devrait apparaitre au centre de votre écran, intitulée \sphinxcode{\sphinxupquote{Information de la carte}}. Copiez le code \sphinxcode{\sphinxupquote{SN}}.

\noindent\sphinxincludegraphics{{fig_info}.png}

\sphinxAtStartPar
Dans le repertoire \sphinxcode{\sphinxupquote{pH\_meter\_V2.0}} contenant l’ensemble des fichiers nécessaires au fonctionnement du programme, ouvrez dans un éditeur de texte, par exemple \sphinxstylestrong{gedit},
le fichier \sphinxcode{\sphinxupquote{lib\_pH.py}}. Dans ce fichier, juste après l’entête \sphinxstylestrong{\# ACCES ARDUINO}, vous allez pouvoir modifier la fonction \sphinxcode{\sphinxupquote{port\_connexion}}:

\begin{sphinxVerbatim}[commandchars=\\\{\}]
\PYG{k}{def}\PYG{+w}{ }\PYG{n+nf}{port\PYGZus{}connexion}\PYG{p}{(}\PYG{n}{br} \PYG{o}{=} \PYG{l+m+mi}{9600} \PYG{p}{,} \PYG{n}{portIN} \PYG{o}{=} \PYG{l+s+s1}{\PYGZsq{}}\PYG{l+s+s1}{\PYGZsq{}}\PYG{p}{)} \PYG{p}{:}
\PYG{l+s+sd}{\PYGZdq{}\PYGZdq{}\PYGZdq{}}
\PYG{l+s+sd}{Établit la connexion au port série.}

\PYG{l+s+sd}{Parameters}
\PYG{l+s+sd}{\PYGZhy{}\PYGZhy{}\PYGZhy{}\PYGZhy{}\PYGZhy{}\PYGZhy{}\PYGZhy{}\PYGZhy{}\PYGZhy{}\PYGZhy{}}
\PYG{l+s+sd}{br : int}
\PYG{l+s+sd}{    Flux de données en baud.}
\PYG{l+s+sd}{portIN : string}
\PYG{l+s+sd}{    Identifiant du port série sur lequel le script doit lire des données.}

\PYG{l+s+sd}{Returns}
\PYG{l+s+sd}{\PYGZhy{}\PYGZhy{}\PYGZhy{}\PYGZhy{}\PYGZhy{}\PYGZhy{}\PYGZhy{}}
\PYG{l+s+sd}{port : string}
\PYG{l+s+sd}{    Identifiant du port série sur lequel le script doit lire des données.}
\PYG{l+s+sd}{s : serial.tools.list\PYGZus{}ports\PYGZus{}common.ListPortInfo / string}
\PYG{l+s+sd}{    Objet Serial sur lequel on peut appliquer des fonctions d\PYGZsq{}ouverture, de lecture et de fermeture du port série affilié. En cas d\PYGZsq{}échec de connexion, \PYGZsq{}s\PYGZsq{} sera une chaîne de caractères \PYGZdq{}erreur\PYGZdq{}.}

\PYG{l+s+sd}{\PYGZdq{}\PYGZdq{}\PYGZdq{}}

\PYG{n}{arduino\PYGZus{}list}\PYG{o}{=}\PYG{p}{[}\PYG{l+s+s1}{\PYGZsq{}}\PYG{l+s+s1}{7513931383135150F0D1}\PYG{l+s+s1}{\PYGZsq{}}\PYG{p}{,}\PYG{l+s+s1}{\PYGZsq{}}\PYG{l+s+s1}{85035323234351504260}\PYG{l+s+s1}{\PYGZsq{}}\PYG{p}{,}\PYG{l+s+s1}{\PYGZsq{}}\PYG{l+s+s1}{85035323234351E09062}\PYG{l+s+s1}{\PYGZsq{}}\PYG{p}{,}\PYG{l+s+s1}{\PYGZsq{}}\PYG{l+s+s1}{75439313737351402252}\PYG{l+s+s1}{\PYGZsq{}}\PYG{p}{,}\PYG{l+s+s1}{\PYGZsq{}}\PYG{l+s+s1}{8503532323435130F142}\PYG{l+s+s1}{\PYGZsq{}}\PYG{p}{,}\PYG{l+s+s1}{\PYGZsq{}}\PYG{l+s+s1}{75330303934351B05162}\PYG{l+s+s1}{\PYGZsq{}}\PYG{p}{]}

\PYG{k}{if} \PYG{n}{portIN} \PYG{o}{==} \PYG{l+s+s1}{\PYGZsq{}}\PYG{l+s+s1}{\PYGZsq{}} \PYG{p}{:}
    \PYG{n}{ports} \PYG{o}{=} \PYG{n+nb}{list}\PYG{p}{(}\PYG{n}{serial}\PYG{o}{.}\PYG{n}{tools}\PYG{o}{.}\PYG{n}{list\PYGZus{}ports}\PYG{o}{.}\PYG{n}{comports}\PYG{p}{(}\PYG{p}{)}\PYG{p}{)}
\PYG{k}{else} \PYG{p}{:}
    \PYG{n}{ports} \PYG{o}{=} \PYG{p}{[}\PYG{n}{portIN}\PYG{p}{]}
\PYG{n}{i} \PYG{o}{=} \PYG{l+m+mi}{0}
\PYG{n}{conn} \PYG{o}{=} \PYG{k+kc}{False}
\PYG{k}{while} \PYG{n}{conn} \PYG{o}{==} \PYG{k+kc}{False} \PYG{p}{:}
    \PYG{k}{try} \PYG{p}{:}
        \PYG{n}{port} \PYG{o}{=} \PYG{n}{ports}\PYG{p}{[}\PYG{n}{i}\PYG{p}{]}
        \PYG{k}{if} \PYG{n}{portIN} \PYG{o}{==} \PYG{l+s+s1}{\PYGZsq{}}\PYG{l+s+s1}{\PYGZsq{}} \PYG{o+ow}{and} \PYG{p}{(}\PYG{n}{port}\PYG{o}{.}\PYG{n}{manufacturer} \PYG{o}{==} \PYG{l+s+s1}{\PYGZsq{}}\PYG{l+s+s1}{Arduino (www.arduino.cc)}\PYG{l+s+s1}{\PYGZsq{}} \PYG{o+ow}{or} \PYG{n}{port}\PYG{o}{.}\PYG{n}{serial\PYGZus{}number} \PYG{o+ow}{in} \PYG{n}{arduino\PYGZus{}list} \PYG{p}{)}\PYG{p}{:}
            \PYG{n}{port} \PYG{o}{=} \PYG{n}{port}\PYG{o}{.}\PYG{n}{device}
            \PYG{n}{port} \PYG{o}{=} \PYG{p}{(}\PYG{n}{port}\PYG{p}{)}\PYG{o}{.}\PYG{n}{replace}\PYG{p}{(}\PYG{l+s+s1}{\PYGZsq{}}\PYG{l+s+s1}{cu}\PYG{l+s+s1}{\PYGZsq{}}\PYG{p}{,}\PYG{l+s+s1}{\PYGZsq{}}\PYG{l+s+s1}{tty}\PYG{l+s+s1}{\PYGZsq{}}\PYG{p}{)}
            \PYG{n}{s} \PYG{o}{=} \PYG{n}{serial}\PYG{o}{.}\PYG{n}{Serial}\PYG{p}{(}\PYG{n}{port}\PYG{o}{=}\PYG{n}{port}\PYG{p}{,} \PYG{n}{baudrate}\PYG{o}{=}\PYG{n}{br}\PYG{p}{,} \PYG{n}{timeout}\PYG{o}{=}\PYG{l+m+mi}{5}\PYG{p}{)}
            \PYG{n}{conn} \PYG{o}{=} \PYG{k+kc}{True}
            \PYG{n+nb}{print}\PYG{p}{(}\PYG{l+s+s1}{\PYGZsq{}}\PYG{l+s+s1}{Connexion établie avec le port}\PYG{l+s+s1}{\PYGZsq{}}\PYG{p}{,} \PYG{n}{port}\PYG{p}{)}
        \PYG{k}{else} \PYG{p}{:}
            \PYG{n}{s} \PYG{o}{=} \PYG{n}{serial}\PYG{o}{.}\PYG{n}{Serial}\PYG{p}{(}\PYG{n}{port}\PYG{o}{=}\PYG{n}{port}\PYG{p}{,} \PYG{n}{baudrate}\PYG{o}{=}\PYG{n}{br}\PYG{p}{,} \PYG{n}{timeout}\PYG{o}{=}\PYG{l+m+mi}{5}\PYG{p}{)}
            \PYG{n}{conn} \PYG{o}{=} \PYG{k+kc}{True}
            \PYG{n+nb}{print}\PYG{p}{(}\PYG{l+s+s1}{\PYGZsq{}}\PYG{l+s+s1}{Connexion établie avec le port}\PYG{l+s+s1}{\PYGZsq{}}\PYG{p}{,} \PYG{n}{port}\PYG{p}{)}
    \PYG{k}{except} \PYG{p}{:}
        \PYG{n}{i} \PYG{o}{+}\PYG{o}{=} \PYG{l+m+mi}{1}
        \PYG{k}{if} \PYG{n}{i} \PYG{o}{\PYGZgt{}}\PYG{o}{=} \PYG{n+nb}{len}\PYG{p}{(}\PYG{n}{ports}\PYG{p}{)} \PYG{p}{:}
            \PYG{n+nb}{print}\PYG{p}{(}\PYG{l+s+s2}{\PYGZdq{}}\PYG{l+s+s2}{/!}\PYG{l+s+s2}{\PYGZbs{}}\PYG{l+s+s2}{ Port de connexion non détecté. Merci de rétablir la connexion non établie : connexion au processeur dans les réglages avant utilisation.}\PYG{l+s+s2}{\PYGZdq{}}\PYG{p}{)}
            \PYG{n}{s} \PYG{o}{=} \PYG{l+s+s1}{\PYGZsq{}}\PYG{l+s+s1}{error}\PYG{l+s+s1}{\PYGZsq{}}
            \PYG{n}{portIN} \PYG{o}{=} \PYG{l+s+s1}{\PYGZsq{}}\PYG{l+s+s1}{\PYGZsq{}}
            \PYG{n}{conn} \PYG{o}{=} \PYG{k+kc}{True}
        \PYG{k}{pass}
\PYG{k}{return} \PYG{n}{port} \PYG{p}{,} \PYG{n}{s}
\end{sphinxVerbatim}

\sphinxAtStartPar
Dans cette fonction, ajouter à la liste \sphinxcode{\sphinxupquote{arduino\_list}} le code SN du nouveau pH\sphinxhyphen{}mètre :

\noindent\sphinxincludegraphics{{fig_list_arduino}.png}

\sphinxAtStartPar
Sauvegardez les modifications et relancez le programme. Le pH\sphinxhyphen{}mètre devrait maintenant correctement se connecter.

\sphinxAtStartPar
Le programme est désormais lancé et le pH\sphinxhyphen{}mètre prêt à être utilisé.


\subsection{La suite}
\label{\detokenize{install:la-suite}}
\sphinxAtStartPar
Il est maintenant temps de passer aux étapes suivantes:

\sphinxAtStartPar
{\hyperref[\detokenize{calib:calibration}]{\sphinxcrossref{\DUrole{std}{\DUrole{std-ref}{Calibration du pH mètre}}}}}

\sphinxAtStartPar
{\hyperref[\detokenize{mesure:mesure}]{\sphinxcrossref{\DUrole{std}{\DUrole{std-ref}{Mesures du pH}}}}}

\sphinxAtStartPar
{\hyperref[\detokenize{graph:graph}]{\sphinxcrossref{\DUrole{std}{\DUrole{std-ref}{Représentation graphique des données}}}}}

\sphinxstepscope


\chapter{Calibration du pH mètre}
\label{\detokenize{calib:calibration-du-ph-metre}}\label{\detokenize{calib:calibration}}\label{\detokenize{calib::doc}}
\sphinxAtStartPar
Pour une meilleure exactitude, un étalonnage fréquent de l’instrument est recommandé. Un étalonnage
est indispensable dans les cas suivants :
\begin{itemize}
\item {} 
\sphinxAtStartPar
L’électrode a été remplacée

\item {} 
\sphinxAtStartPar
Au moins une fois par mois

\item {} 
\sphinxAtStartPar
Après avoir mesuré des produits chimiques agressifs

\item {} 
\sphinxAtStartPar
Lorsqu’une grande exactitude est requise

\end{itemize}


\section{Préparation}
\label{\detokenize{calib:preparation}}
\sphinxAtStartPar
Versez une petite quantité de solution \sphinxstylestrong{pH 7,00} et \sphinxstylestrong{pH 4,00} dans deux récipients propres. Pour une
meilleure exactitude, il est conseillé de prendre deux récipients par solution, un récipient pour le rinçage et un autre pour
l’étalonnage à proprement dit. Le choix des solutions étalons se fait selon l’utilisation, soit \sphinxstylestrong{pH 7,00} et
\sphinxstylestrong{pH 4,00} dans le cas d’une calibration à deux tampons et \sphinxstylestrong{pH 10,00}, \sphinxstylestrong{pH 7,00} et \sphinxstylestrong{pH 4,00} dans le cas d’une
calibration à trois tampons.


\section{Procédure}
\label{\detokenize{calib:procedure}}

\subsection{Calibration à deux solutions tampons}
\label{\detokenize{calib:calibration-a-deux-solutions-tampons}}
\sphinxAtStartPar
Ôtez le capuchon de protection en bout de la sonde pH, \sphinxstylestrong{rincez} celle\sphinxhyphen{}ci avec de l’eau distillée puis immergez l’électrode
dans une solution \sphinxstylestrong{pH 7,00} ; agitez délicatement pendant quelques secondes ; immergez la sonde de température
et \sphinxstylestrong{attendez 2 à 3 minutes} pour avoir un équilibre thermique correct.

\sphinxAtStartPar
Choisissez dans le menu interface l’option \sphinxcode{\sphinxupquote{Calibrer}} en appuyant sur le chiffre \sphinxcode{\sphinxupquote{1}} et en validant avec
la touche \sphinxstylestrong{Entrée}:

\begin{sphinxVerbatim}[commandchars=\\\{\}]
===========================================================================
MENU PRINCIPAL
===========================================================================
Que souhaitez\PYGZhy{}vous faire ?
1 \PYGZhy{} Calibrer
2 \PYGZhy{} Mesurer
3 \PYGZhy{} Représenter graphiquement
4 \PYGZhy{} Quitter
===========================================================================
?

\PYGZhy{}\PYGZgt{} 1

    ===========================================================================
    MENU CALIBRATION
    ==========================================================================
    Voulez \PYGZhy{} vous :
    1 \PYGZhy{} Calibrer avec deux tampons (pH 7 et 4) ?
    2 \PYGZhy{} Calibrer avec trois tampons (pH 7, 4 et 10) ?
    3 \PYGZhy{} Calibrer à partir d\PYGZsq{}une calibration déjà existante dans le répertoire
    4 \PYGZhy{} Quitter le menu calibration et retourner au menu principal
    ===========================================================================
    ?
\end{sphinxVerbatim}

\sphinxAtStartPar
Sélectionnez ensuite l’option \sphinxcode{\sphinxupquote{Calibrer avec deux tampons}} de la même façon.
Validez une seconde fois lorsque vous serez prêt à mesurer le tampon à
\sphinxstylestrong{pH 7,00}. Le programme vous demandera alors de patienter \sphinxstylestrong{une minute} le temps que la sonde pH se
stabilise puis effectuera \sphinxstylestrong{100 mesures} pour une durée approximative de \sphinxstylestrong{deux minutes}.

\sphinxAtStartPar
Une fois les mesures à \sphinxstylestrong{pH 7,00} effectuées, \sphinxstylestrong{rincez} la sonde pH avec de l’eau distillée.

\sphinxAtStartPar
Plongez l’électrode dans la solution \sphinxstylestrong{pH 4,00} et attendez quelques minutes pour la stabilisation de la
mesure. Une fois prêt à mesurer le tampon à pH 4,00, appuyez sur la touche \sphinxstylestrong{Entrée} pour initier les
mesures. Comme plus tôt, le programme vous demandera alors de patienter \sphinxstylestrong{1 minute} le temps que la sonde pH
se stabilise puis effectuera ses \sphinxstylestrong{100 mesures}.

\sphinxAtStartPar
Une fois les mesures effectuées, le programme vous demandera:

\begin{sphinxVerbatim}[commandchars=\\\{\}]
’Voulez\PYGZhy{}vous visualiser la calibration (O/N) ?’
\end{sphinxVerbatim}

\sphinxAtStartPar
En répondant \sphinxcode{\sphinxupquote{O}}, \sphinxcode{\sphinxupquote{o}}, \sphinxcode{\sphinxupquote{Y}} ou \sphinxcode{\sphinxupquote{y}}, vous obtiendrez alors la courbe d’étalonnage de la calibration
que vous venez d’effectuer, ses paramètres et le coefficient de corrélation. L’étalonnage est alors terminé, vous pouvez passer
à vos {\hyperref[\detokenize{mesure:mesure}]{\sphinxcrossref{\DUrole{std}{\DUrole{std-ref}{Mesures du pH}}}}}.

\sphinxAtStartPar
Si le programme juge que la calibration n’est pas satisfaisante \sphinxstylestrong{(R2 \textless{} 0.95)}, il vous sera proposé de
recalibrer votre pH\sphinxhyphen{}mètre.
Si vous souhaitez recalibrer le pH\sphinxhyphen{}mètre, choisissez l’option \sphinxcode{\sphinxupquote{1}}.
Si vous souhaitez continuer ainsi et commencer à mesurer, choisissez l’option \sphinxcode{\sphinxupquote{2}}.


\subsection{Calibration à trois solution tampons}
\label{\detokenize{calib:calibration-a-trois-solution-tampons}}
\sphinxAtStartPar
Si vous souhaitez calibrer votre pH\sphinxhyphen{}mètre avec \sphinxstylestrong{3 solutions tampons} la procédure est la même mais il vous faudra
continuer avec la solution \sphinxstylestrong{pH 10,00}.
Pour cela, choisissez dans l’interface \sphinxcode{\sphinxupquote{MENU CALIBRATION}} l’option \sphinxcode{\sphinxupquote{2 \sphinxhyphen{} Calibrer avec trois
tampons}}:

\begin{sphinxVerbatim}[commandchars=\\\{\}]
===========================================================================
MENU CALIBRATION
==========================================================================
Voulez \PYGZhy{} vous :
1 \PYGZhy{} Calibrer avec deux tampons (pH 7 et 4) ?
2 \PYGZhy{} Calibrer avec trois tampons (pH 7, 4 et 10) ?
3 \PYGZhy{} Calibrer à partir d\PYGZsq{}une calibration déjà existante dans le répertoire
4 \PYGZhy{} Quitter le menu calibration et retourner au menu principal
===========================================================================
?

\PYGZhy{}\PYGZgt{} 2

Quel est le pH de la troisième solution que vous souhaitez utilisez ? ( 9 / 10 / 11)

\PYGZhy{}\PYGZgt{} 10

Prêt pour calibration pH7 ?

Patientez 1 min le temps que la sonde se stabilise

Les mesures commencent

La température relevée est de 25.91732608695652

Prêt pour calibration pH4 ?

Patientez 1 min le temps que la sonde se stabilise

Les mesures commencent

La température relevée est de 25.974727592267133

Prêt pour calibration pH10 ?

Patientez 1 min le temps que la sonde se stabilise

Les mesures commencent

La température relevée est de 26.36460258780037

Les paramètres a et b de notre regression linéaire sont [\PYGZhy{}2.91457992e\PYGZhy{}02  3.09806410e+01]
(\PYGZsq{}r2 =\PYGZsq{}, 0.9974)

Voulez\PYGZhy{}vous visualiser la calibration (O/N) ?
\end{sphinxVerbatim}

\sphinxAtStartPar
Les données de la calibration effectuée sont enregistrées dans le dossier \sphinxcode{\sphinxupquote{CALIB}}.

\begin{sphinxadmonition}{note}{Note:}
\sphinxAtStartPar
Si vous ne possédez pas de solution tampon pH 10 la calibration à 3 solutions est aussi possible avec des pH 9 et pH 11. Le programme vous demande pour cela:

\begin{sphinxVerbatim}[commandchars=\\\{\}]
Quel est le pH de la troisième solution que vous souhaitez utilisez ? ( 9 / 10 / 11)
\end{sphinxVerbatim}

\sphinxAtStartPar
Il suffit alors de renseigner la valeur pH de la troisième solution tampon utilisée et la procédure est la même que décrite plus haut.
\end{sphinxadmonition}


\subsection{Calibration à partir de données enregistrées}
\label{\detokenize{calib:calibration-a-partir-de-donnees-enregistrees}}
\sphinxAtStartPar
Il est aussi possible de calibrer votre pH\sphinxhyphen{}mètre à partir de données de précédentes calibrations déja répertoriées dans le dossier \sphinxcode{\sphinxupquote{CALIB}}.
Pour cela, choisissez dans le \sphinxcode{\sphinxupquote{MENU CALIBRATION}} l’option \sphinxcode{\sphinxupquote{3 \sphinxhyphen{} Calibration à partir d’une calibration déjà exitante dans le répertoire}}.
Une liste de fichier vous est proposée de la forme:

\begin{sphinxVerbatim}[commandchars=\\\{\}]
===========================================================================
MENU CALIBRATION
==========================================================================
Voulez \PYGZhy{} vous :
1 \PYGZhy{} Calibrer avec deux tampons (pH 7 et 4) ?
2 \PYGZhy{} Calibrer avec trois tampons (pH 7, 4 et 10) ?
3 \PYGZhy{} Calibrer à partir d\PYGZsq{}une calibration déjà existante dans le répertoire
4 \PYGZhy{} Quitter le menu calibration et retourner au menu principal
===========================================================================
?

\PYGZhy{}\PYGZgt{} 3

Cette option n\PYGZsq{}est possible que pour des calibrations à 3 solutions, quel est le pH de la troisième solution de la calibration que vous souhaitez utilisez ? ( 9 / 10 / 11)

\PYGZhy{}\PYGZgt{} 10

Calibrations disponibles:
0 \PYGZhy{} ./CALIB/fichier\PYGZus{}calibration\PYGZus{}pH10.0 Mon Jun 23 11:46:13 2025.csv

Choisissez votre calibration en entrant son numéro d\PYGZsq{}ordre:

\PYGZhy{}\PYGZgt{} 0

Mon Jun 23 11:46:13 2025
./CALIB/fichier\PYGZus{}calibration\PYGZus{}pH4.01 Mon Jun 23 11:46:13 2025.csv
./CALIB/fichier\PYGZus{}calibration\PYGZus{}pH7.01 Mon Jun 23 11:46:13 2025.csv
./CALIB/fichier\PYGZus{}calibration\PYGZus{}pH10.0 Mon Jun 23 11:46:13 2025.csv

Les paramètres a et b de notre regression linéaire sont [\PYGZhy{}2.91457992e\PYGZhy{}02  3.09806410e+01]
(\PYGZsq{}r2 =\PYGZsq{}, 0.9974)
\end{sphinxVerbatim}

\sphinxAtStartPar
Il suffit alors de renseigner le numéro d’ordre des données que vous souhaitez utiliser et valider avec la touche \sphinxstylestrong{Entrée}.

\begin{sphinxadmonition}{note}{Note:}
\sphinxAtStartPar
Le fichier à sélectionner est de la forme \sphinxcode{\sphinxupquote{./CALIB/fichier\_calibration\_pH10.00 Day Month H\_min\_sec Year.csv}} mais implicitement
ceux correspondant à la même calibration pour les \sphinxstylestrong{pH 4.00} et \sphinxstylestrong{7.00} vont aussi être utilisé. Notez que si la calibration que vos souhaitez
retrouver n’utilisait pas de solutions pH 10 mais pH 9 ou pH 11 le programme vous demande avant de lister les fichiers:

\begin{sphinxVerbatim}[commandchars=\\\{\}]
\PYG{n}{Cette} \PYG{n}{option} \PYG{n}{n}\PYG{l+s+s1}{\PYGZsq{}}\PYG{l+s+s1}{est possible que pour des calibrations à 3 solutions, quel est le pH de la troisième solution de la calibration que vous souhaitez utilisez ? ( 9 / 10 / 11)}
\end{sphinxVerbatim}

\sphinxAtStartPar
Il vous suffit d’indiquer le pH de la troisième solution, tout les fichiers de calibration utilisant la solution au pH indiqué vont être listés. Une fois une
calibration sélectionnée les fichiers correspondant à la même calibration pour les \sphinxstylestrong{pH 4.00} et \sphinxstylestrong{7.00} le seront aussi implicitement.
\end{sphinxadmonition}

\sphinxAtStartPar
Une fois le fichier renseigné la calibration sera effectuée, la droite de la calibration choisie apparaitra à
l’écran ainsi que la courbe de l’évolution de l’écart\sphinxhyphen{}type des mesures au cours du temps.

\noindent\sphinxincludegraphics{{fig_calib}.png}

\sphinxAtStartPar
Vous pouvez maintenant effectuer vos {\hyperref[\detokenize{mesure:mesure}]{\sphinxcrossref{\DUrole{std}{\DUrole{std-ref}{Mesures du pH}}}}}.

\begin{sphinxadmonition}{warning}{Avertissement:}
\sphinxAtStartPar
Assurez\sphinxhyphen{}vous que les fichiers que vous renseignez correspondent au format demandé :
un fichier \sphinxcode{\sphinxupquote{csv}}, avec pour chaque ligne les informations \sphinxcode{\sphinxupquote{temps(s);température(°C);voltage(mV)}}:

\begin{sphinxVerbatim}[commandchars=\\\{\}]
\PYG{l+m+mf}{0.00}\PYG{p}{;} \PYG{l+m+mf}{23.27}\PYG{p}{;} \PYG{l+m+mf}{180.00}
\PYG{l+m+mf}{0.25}\PYG{p}{;} \PYG{l+m+mf}{23.33}\PYG{p}{;} \PYG{l+m+mf}{180.00}
\PYG{l+m+mf}{0.51}\PYG{p}{;} \PYG{l+m+mf}{23.33}\PYG{p}{;} \PYG{l+m+mf}{180.00}
\PYG{l+m+mf}{0.76}\PYG{p}{;} \PYG{l+m+mf}{23.33}\PYG{p}{;} \PYG{l+m+mf}{180.00}
\PYG{l+m+mf}{1.02}\PYG{p}{;} \PYG{l+m+mf}{23.37}\PYG{p}{;} \PYG{l+m+mf}{180.00}
\PYG{l+m+mf}{1.27}\PYG{p}{;} \PYG{l+m+mf}{23.40}\PYG{p}{;} \PYG{l+m+mf}{180.00}
\PYG{l+m+mf}{1.52}\PYG{p}{;} \PYG{l+m+mf}{23.43}\PYG{p}{;} \PYG{l+m+mf}{180.00}
\PYG{l+m+mf}{1.78}\PYG{p}{;} \PYG{l+m+mf}{23.43}\PYG{p}{;} \PYG{l+m+mf}{179.00}
\PYG{l+m+mf}{2.03}\PYG{p}{;} \PYG{l+m+mf}{23.47}\PYG{p}{;} \PYG{l+m+mf}{180.00}
\PYG{o}{.}\PYG{o}{.}\PYG{o}{.}
\end{sphinxVerbatim}

\sphinxAtStartPar
Par défaut cela correspond au format des données de {\hyperref[\detokenize{mesure:mesure}]{\sphinxcrossref{\DUrole{std}{\DUrole{std-ref}{Mesures du pH}}}}}.
\end{sphinxadmonition}

\sphinxstepscope


\chapter{Mesures du pH}
\label{\detokenize{mesure:mesures-du-ph}}\label{\detokenize{mesure:mesure}}\label{\detokenize{mesure::doc}}
\begin{sphinxadmonition}{warning}{Avertissement:}
\sphinxAtStartPar
Il est important que la solution à mesurer soit dans les mêmes conditions de pression mais surtout de
température que les solutions tampons lors de la calibration, puisque la correction du pH en fonction de
la température s’effectue pendant la calibration. Si la température de votre solution évolue au cours du
temps une nouvelle calibration est nécessaire.
\end{sphinxadmonition}


\section{Préparation}
\label{\detokenize{mesure:preparation}}
\sphinxAtStartPar
Après avoir oté le capuchon de protection, plongez la sonde pH ainsi que la sonde de température dans la solution à mesurer.
Agitez l’électrode pendant quelques secondes puis stabilisez\sphinxhyphen{}la.

\sphinxAtStartPar
Sélectionnez dans le \sphinxcode{\sphinxupquote{MENU PRINCIPAL}} l’option \sphinxcode{\sphinxupquote{2 \sphinxhyphen{} Mesurer}} en appuyant sur la touche \sphinxcode{\sphinxupquote{2}} de votre clavier
puis valider avec la touche \sphinxstylestrong{Entrée}.
Le programme vous demande alors si vous êtes prêt à initier les mesures, appuyez sur \sphinxstylestrong{Entrée} lorsque vous le serez.

\begin{sphinxadmonition}{note}{Note:}
\sphinxAtStartPar
Pour avoir des mesures précises, il est nécessaire que l’instrument ait été étalonné au préalable. Si les mesures sont effectuées dans des échantillons
successifs, il est recommandé de rincer l’électrode entre chaque échantillon, afin de ne pas contaminer
les échantillons entre\sphinxhyphen{}eux.
\end{sphinxadmonition}


\section{Prise de mesure}
\label{\detokenize{mesure:prise-de-mesure}}
\sphinxAtStartPar
Une fois l’opération lancée, \sphinxstylestrong{10} valeurs de pH sont mesurées, pour ces 10 valeurs le programme va afficher
le \sphinxstylestrong{temps} (en secondes) de la prise mesure par rapport à l’initiation des mesures, la \sphinxstylestrong{température moyenne}
et \sphinxstylestrong{l’écart\sphinxhyphen{}type des mesures de température}, le \sphinxstylestrong{voltage moyen}, \sphinxstylestrong{l’écart\sphinxhyphen{}type des mesures de voltage}, le \sphinxstylestrong{pH moyen} mesuré, \sphinxstylestrong{l’écart\sphinxhyphen{}type
des mesures de pH} et enfin une valeur de la \sphinxstylestrong{stabilité}:

\begin{sphinxVerbatim}[commandchars=\\\{\}]
===========================================================================
MENU PRINCIPAL
===========================================================================
Que souhaitez\PYGZhy{}vous faire ?
1 \PYGZhy{} Calibrer
2 \PYGZhy{} Mesurer
3 \PYGZhy{} Représenter graphiquement
4 \PYGZhy{} Quitter
===========================================================================
?

\PYGZhy{}\PYGZgt{} 2

Prêt à mesurer ?

Les mesures commencent

Temps: 0.00
Température: 25.54 +/\PYGZhy{} 0.02
Voltage: 760.00 +/\PYGZhy{} 2.86
PH: 8.83 +/\PYGZhy{} 0.08
Stabilité: 2.14

Temps: 2.65
Température: 25.55 +/\PYGZhy{} 0.02
Voltage: 761.64 +/\PYGZhy{} 1.67
PH: 8.78 +/\PYGZhy{} 0.05
Stabilité: 2.94
\end{sphinxVerbatim}

\sphinxAtStartPar
La stabilité est un coefficient qui représente l’ecart entre \sphinxstylestrong{10} valeurs de pH mesurées et les \sphinxstylestrong{10} valeurs mesurées précedentes.
La mesure de la sonde est considérée comme stabilisée lorque que le coeffcient de stabilité est proche de \sphinxstylestrong{zéro}.
Tant que la valeur du coeffcient de stabilité n’est pas satisfaisante le programme continue d’effectuer des mesures.

\sphinxAtStartPar
Une fois une stabilité satisfaisante atteinte, le programme arrête ses mesures et demande à l’utilisateur s’il souhaite continuer ou non la prise de mesures.
L’utilisateur a alors le choix de poursuivre les mesures en répondant par \sphinxcode{\sphinxupquote{O}}, \sphinxcode{\sphinxupquote{o}}, \sphinxcode{\sphinxupquote{Y}} ou \sphinxcode{\sphinxupquote{y}}. Le programme effectue alors 20 mesures
supplémentaires avant de reproposer de continuer ou non.

\sphinxAtStartPar
Si l’utilisateur juge que le nombre de mesures est suffisant et souhaite s’arrêter là, il répond alors à la question posée avec \sphinxcode{\sphinxupquote{N}} ou \sphinxcode{\sphinxupquote{n}}.
Le programme propose alors à l’utilisateur s’il souhaite voir la {\hyperref[\detokenize{graph:graph}]{\sphinxcrossref{\DUrole{std}{\DUrole{std-ref}{Représentation graphique des données}}}}} qu’il vient de mesurer.
L’ensemble des mesures est enregistré dans un fichier \sphinxcode{\sphinxupquote{csv}} de la forme \sphinxcode{\sphinxupquote{fichier mesure Day Month H\_min\_sec Year.csv}} dans le dossier \sphinxcode{\sphinxupquote{DATA}}.

\begin{sphinxadmonition}{warning}{Avertissement:}
\sphinxAtStartPar
Si vous souhaitez enregistrer l’évolution des valeurs de pH de vos données en fonction du temps référez\sphinxhyphen{}vous maintenant à la prochaine étape :

\sphinxAtStartPar
{\hyperref[\detokenize{graph:graph}]{\sphinxcrossref{\DUrole{std}{\DUrole{std-ref}{Représentation graphique des données}}}}}
\end{sphinxadmonition}


\section{Mesures d’Alcalinité}
\label{\detokenize{mesure:mesures-d-alcalinite}}
\sphinxAtStartPar
Si vous souhaitez utilisez ces pH mètre pour une mesure d’alcalinité la procédure est la même mais poursuivez les mesures jusqu’à la fin de votre titrage avec l’option

\begin{sphinxVerbatim}[commandchars=\\\{\}]
Continuer les mesures (O/N) ?

\PYGZhy{}\PYGZgt{} N
\end{sphinxVerbatim}

\sphinxAtStartPar
Le titrage peut être considéré comme terminé une fois la valeur seuil de pH atteinte à \sphinxstylestrong{pH 3.7}.

\sphinxAtStartPar
Assurez vous que la mesure de pH soit bien stabilisée avant d’ajouter un nouvel incrément de \sphinxstylestrong{HCl}.

\sphinxAtStartPar
La représentation graphique de vos mesure aux termes du titrage ainsi que le fichier contenant les données
mesurées, enregistré dans le dossier \sphinxcode{\sphinxupquote{DATA}}, vous permettront de déterminer l’alcalinité à partir des méthodes de
\sphinxstylestrong{Gran} et de \sphinxstylestrong{Culberson}.

\sphinxstepscope


\chapter{Représentation graphique des données}
\label{\detokenize{graph:representation-graphique-des-donnees}}\label{\detokenize{graph:graph}}\label{\detokenize{graph::doc}}
\begin{sphinxadmonition}{warning}{Avertissement:}
\sphinxAtStartPar
Que ce soit à l’issue de votre prise de vos {\hyperref[\detokenize{mesure:mesure}]{\sphinxcrossref{\DUrole{std}{\DUrole{std-ref}{Mesures du pH}}}}} ou pour des données plus anciennes,
lorsque ce vous souhaitez enregistrer votre figure, ne fermer la fenêtre affichée par le programme la contenant qu’une fois avoir sélectionné
l’option d’enregistrement:

\begin{sphinxVerbatim}[commandchars=\\\{\}]
Sauver (O/N) ?

\PYGZhy{}\PYGZgt{} O
\end{sphinxVerbatim}

\sphinxAtStartPar
Sinon vous n’enregistrerez qu’une page blanche.
\end{sphinxadmonition}

\sphinxAtStartPar
Ce programme permet de représenter et d’enregistrer graphiquement les mesures
effectuées juste à l’instant ou lors de prises de mesures plus anciennes.
Selectionnez pour cela l’option \sphinxcode{\sphinxupquote{3 \sphinxhyphen{} Représenter graphiquement}} dans le \sphinxcode{\sphinxupquote{MENU PRINCIPAL}}.

\sphinxAtStartPar
Le programme vous affichera la liste des données de mesures disponibles dans le dossier \sphinxcode{\sphinxupquote{DATA}}:

\begin{sphinxVerbatim}[commandchars=\\\{\}]
===========================================================================
MENU PRINCIPAL
===========================================================================
Que souhaitez\PYGZhy{}vous faire ?
1 \PYGZhy{} Calibrer
2 \PYGZhy{} Mesurer
3 \PYGZhy{} Représenter graphiquement
4 \PYGZhy{} Quitter
===========================================================================
?

\PYGZhy{}\PYGZgt{} 3

    fichiers disponibles:
    0 \PYGZhy{} ./DATA/fichier\PYGZus{}mesure Thu Jun  6 12\PYGZus{}58\PYGZus{}56 2024.csv
    1 \PYGZhy{} ./DATA/fichier\PYGZus{}mesure Thu Jun  6 12\PYGZus{}03\PYGZus{}47 2024.csv
    2 \PYGZhy{} ./DATA/fichier\PYGZus{}mesure Thu Jun  6 12\PYGZus{}52\PYGZus{}02 2024.csv
    3 \PYGZhy{} ./DATA/fichier\PYGZus{}mesure Thu Jun  6 12\PYGZus{}04\PYGZus{}58 2024.csv
    4 \PYGZhy{} ./DATA/fichier\PYGZus{}mesure Thu Jun  6 12\PYGZus{}14\PYGZus{}54 2024.csv
    5 \PYGZhy{} ./DATA/fichier\PYGZus{}mesure Fri Jun 28 11\PYGZus{}19\PYGZus{}49 2024.csv
    6 \PYGZhy{} ./DATA/fichier\PYGZus{}mesure Fri Jun 28 11\PYGZus{}27\PYGZus{}36 2024.csv
    7 \PYGZhy{} ./DATA/fichier\PYGZus{}mesure Thu Jun  6 12\PYGZus{}46\PYGZus{}56 2024.csv
    8 \PYGZhy{} ./DATA/fichier\PYGZus{}mesure Fri Jun 28 11\PYGZus{}21\PYGZus{}37 2024.csv
    ...
\end{sphinxVerbatim}

\sphinxAtStartPar
Il suffit alors de renseigner le numéro d’ordre des données que vous souhaitez utiliser et valider avec la touche \sphinxstylestrong{Entrée}.

\noindent\sphinxincludegraphics{{fig_mesure}.png}

\sphinxAtStartPar
Le graphique de vos données va s’afficher. Le programme va alors vous demander:

\begin{sphinxVerbatim}[commandchars=\\\{\}]
Sauver (O/N) ?
\end{sphinxVerbatim}

\sphinxAtStartPar
Il est possible si vous le souhaitez d’enregistrer le graphique obtenu dans le dossier \sphinxcode{\sphinxupquote{FIGURES}}, il vous suffit alors de répondre avec les touches
\sphinxcode{\sphinxupquote{O}}, \sphinxcode{\sphinxupquote{o}}, \sphinxcode{\sphinxupquote{Y}} ou \sphinxcode{\sphinxupquote{y}}.
Le fichier sera alors enregistré au format \sphinxcode{\sphinxupquote{pdf}} dans le dossier \sphinxcode{\sphinxupquote{FIGURES}}.

\sphinxstepscope


\chapter{Documentation des fonctions}
\label{\detokenize{library:module-lib_pH}}\label{\detokenize{library:documentation-des-fonctions}}\label{\detokenize{library::doc}}\index{module@\spxentry{module}!lib\_pH@\spxentry{lib\_pH}}\index{lib\_pH@\spxentry{lib\_pH}!module@\spxentry{module}}
\sphinxAtStartPar
Created on Wed May 22 16:09:59 2024

\sphinxAtStartPar
@author ori: Clathi
\index{Calibration() (dans le module lib\_pH)@\spxentry{Calibration()}\spxextra{dans le module lib\_pH}}

\begin{fulllineitems}
\phantomsection\label{\detokenize{library:lib_pH.Calibration}}
\pysigstartsignatures
\pysiglinewithargsret
{\sphinxcode{\sphinxupquote{lib\_pH.}}\sphinxbfcode{\sphinxupquote{Calibration}}}
{\sphinxparam{\DUrole{n}{calib3}}\sphinxparamcomma \sphinxparam{\DUrole{n}{buffer\_value}}\sphinxparamcomma \sphinxparam{\DUrole{n}{buffers}\DUrole{o}{=}\DUrole{default_value}{{[}7, 4{]}}}\sphinxparamcomma \sphinxparam{\DUrole{n}{n}\DUrole{o}{=}\DUrole{default_value}{200}}\sphinxparamcomma \sphinxparam{\DUrole{n}{port\_test}\DUrole{o}{=}\DUrole{default_value}{\textquotesingle{}\textquotesingle{}}}}
{}
\pysigstopsignatures
\sphinxAtStartPar
Calibre la sonde pH pour 2 et 3 tampons (7, 4 et 10) en 100 mesures. Corrige les valeurs obtenues en fonction de la température.
\begin{quote}\begin{description}
\sphinxlineitem{Paramètres}\begin{itemize}
\item {} 
\sphinxAtStartPar
\sphinxstyleliteralstrong{\sphinxupquote{buffers}} (\sphinxstyleliteralemphasis{\sphinxupquote{list}}\sphinxstyleliteralemphasis{\sphinxupquote{, }}\sphinxstyleliteralemphasis{\sphinxupquote{liste}}) \textendash{} DESCRIPTION. The default is {[}7, 4{]}.

\item {} 
\sphinxAtStartPar
\sphinxstyleliteralstrong{\sphinxupquote{n}} (\sphinxstyleliteralemphasis{\sphinxupquote{int}}\sphinxstyleliteralemphasis{\sphinxupquote{, }}\sphinxstyleliteralemphasis{\sphinxupquote{nombre de mesures}}) \textendash{} DESCRIPTION. The default is 100.

\end{itemize}

\sphinxlineitem{Renvoie}
\sphinxAtStartPar
\sphinxstylestrong{model} \textendash{} DESCRIPTION. Les paramètres a et b de la courbe de calibration, a correspond au coefficient directeur et b à l’ordonnée à l’origine.

\sphinxlineitem{Type renvoyé}
\sphinxAtStartPar
list, liste

\end{description}\end{quote}

\end{fulllineitems}

\index{Calibration\_existante() (dans le module lib\_pH)@\spxentry{Calibration\_existante()}\spxextra{dans le module lib\_pH}}

\begin{fulllineitems}
\phantomsection\label{\detokenize{library:lib_pH.Calibration_existante}}
\pysigstartsignatures
\pysiglinewithargsret
{\sphinxcode{\sphinxupquote{lib\_pH.}}\sphinxbfcode{\sphinxupquote{Calibration\_existante}}}
{\sphinxparam{\DUrole{n}{buffer\_existant}}}
{}
\pysigstopsignatures
\sphinxAtStartPar
Calibre la sonde pH pour 3 tampons (7, 4 et 10) à partir d’une calibration déjà existante, et présente dans le même répertoire que ce programme.
\begin{quote}\begin{description}
\sphinxlineitem{Renvoie}
\sphinxAtStartPar
\sphinxstylestrong{model} \textendash{} DESCRIPTION. Les paramètres a et b de la courbe de calibration, a correspond au coefficient directeur et b à l’ordonnée à l’origine.

\sphinxlineitem{Type renvoyé}
\sphinxAtStartPar
list, liste

\end{description}\end{quote}

\end{fulllineitems}

\index{default\_Calibration() (dans le module lib\_pH)@\spxentry{default\_Calibration()}\spxextra{dans le module lib\_pH}}

\begin{fulllineitems}
\phantomsection\label{\detokenize{library:lib_pH.default_Calibration}}
\pysigstartsignatures
\pysiglinewithargsret
{\sphinxcode{\sphinxupquote{lib\_pH.}}\sphinxbfcode{\sphinxupquote{default\_Calibration}}}
{}
{}
\pysigstopsignatures
\sphinxAtStartPar
Calibration par défaut de la sonde pH, effectuée en laboratoire.
\begin{quote}\begin{description}
\sphinxlineitem{Renvoie}
\sphinxAtStartPar
\sphinxstylestrong{model} \textendash{} DESCRIPTION. Les paramètres a et b de la courbe de calibration par défaut, a correspond au coefficient directeur et b à l’ordonnée à l’origine.

\sphinxlineitem{Type renvoyé}
\sphinxAtStartPar
list, liste

\end{description}\end{quote}

\end{fulllineitems}

\index{fn\_settings() (dans le module lib\_pH)@\spxentry{fn\_settings()}\spxextra{dans le module lib\_pH}}

\begin{fulllineitems}
\phantomsection\label{\detokenize{library:lib_pH.fn_settings}}
\pysigstartsignatures
\pysiglinewithargsret
{\sphinxcode{\sphinxupquote{lib\_pH.}}\sphinxbfcode{\sphinxupquote{fn\_settings}}}
{\sphinxparam{\DUrole{n}{portIN}}\sphinxparamcomma \sphinxparam{\DUrole{n}{s}}\sphinxparamcomma \sphinxparam{\DUrole{n}{br}}\sphinxparamcomma \sphinxparam{\DUrole{n}{nb\_inter}}\sphinxparamcomma \sphinxparam{\DUrole{n}{time\_inter}}}
{}
\pysigstopsignatures
\sphinxAtStartPar
Configuration des paramètres modifiables par l’utilisateur.
\begin{quote}\begin{description}
\sphinxlineitem{Paramètres}\begin{itemize}
\item {} 
\sphinxAtStartPar
\sphinxstyleliteralstrong{\sphinxupquote{portIN}} (\sphinxstyleliteralemphasis{\sphinxupquote{string}}) \textendash{} Identifiant du port série sur lequel le script doit lire des données.

\item {} 
\sphinxAtStartPar
\sphinxstyleliteralstrong{\sphinxupquote{s}} (\sphinxstyleliteralemphasis{\sphinxupquote{serial.tools.list\_ports\_common.ListPortInfo}}) \textendash{} Objet Serial sur lequel on peut appliquer des fonctions d’ouverture, de lecture et de fermeture du port série affilié.

\item {} 
\sphinxAtStartPar
\sphinxstyleliteralstrong{\sphinxupquote{br}} (\sphinxstyleliteralemphasis{\sphinxupquote{int}}) \textendash{} Flux de données en baud.

\item {} 
\sphinxAtStartPar
\sphinxstyleliteralstrong{\sphinxupquote{nb\_inter}} (\sphinxstyleliteralemphasis{\sphinxupquote{int}}) \textendash{} Nombre de valeurs utilisées pour constituer une mesure (une mesure correspond à la moyenne de toutes les valeurs prélevées).

\item {} 
\sphinxAtStartPar
\sphinxstyleliteralstrong{\sphinxupquote{time\_inter}} (\sphinxstyleliteralemphasis{\sphinxupquote{float}}) \textendash{} Temps d’intervalle entre chaque prélèvement de valeur au sein d’une mesure.

\end{itemize}

\sphinxlineitem{Renvoie}
\sphinxAtStartPar
\begin{itemize}
\item {} 
\sphinxAtStartPar
\sphinxstylestrong{portIN} (\sphinxstyleemphasis{string}) \textendash{} Identifiant du port série sur lequel le script doit lire des données.

\item {} 
\sphinxAtStartPar
\sphinxstylestrong{s} (\sphinxstyleemphasis{serial.tools.list\_ports\_common.ListPortInfo}) \textendash{} Objet Serial sur lequel on peut appliquer des fonctions d’ouverture, de lecture et de fermeture du port série affilié.

\item {} 
\sphinxAtStartPar
\sphinxstylestrong{br} (\sphinxstyleemphasis{int}) \textendash{} Flux de données en baud.

\item {} 
\sphinxAtStartPar
\sphinxstylestrong{nb\_inter} (\sphinxstyleemphasis{int}) \textendash{} Nombre de valeurs utilisées pour constituer une mesure (une mesure correspond à la moyenne de toutes les valeurs prélevées).

\item {} 
\sphinxAtStartPar
\sphinxstylestrong{time\_inter} (\sphinxstyleemphasis{float}) \textendash{} Temps d’intervalle entre chaque prélèvement de valeur au sein d’une mesure.

\end{itemize}


\end{description}\end{quote}

\end{fulllineitems}

\index{graph() (dans le module lib\_pH)@\spxentry{graph()}\spxextra{dans le module lib\_pH}}

\begin{fulllineitems}
\phantomsection\label{\detokenize{library:lib_pH.graph}}
\pysigstartsignatures
\pysiglinewithargsret
{\sphinxcode{\sphinxupquote{lib\_pH.}}\sphinxbfcode{\sphinxupquote{graph}}}
{}
{}
\pysigstopsignatures
\sphinxAtStartPar
Fait un graphique
ph =f(t) et T=f(t) avec barres d’erreurs
à partir d’un fichier de mesures
sélectionné dans le dossier ./DATA

\sphinxAtStartPar
propose la sauvegarde du fichier dans le dossier ./FIGURES au format pdf

\end{fulllineitems}

\index{indiv\_measure() (dans le module lib\_pH)@\spxentry{indiv\_measure()}\spxextra{dans le module lib\_pH}}

\begin{fulllineitems}
\phantomsection\label{\detokenize{library:lib_pH.indiv_measure}}
\pysigstartsignatures
\pysiglinewithargsret
{\sphinxcode{\sphinxupquote{lib\_pH.}}\sphinxbfcode{\sphinxupquote{indiv\_measure}}}
{\sphinxparam{\DUrole{n}{port\_test}}\sphinxparamcomma \sphinxparam{\DUrole{n}{model}}\sphinxparamcomma \sphinxparam{\DUrole{n}{n}\DUrole{o}{=}\DUrole{default_value}{10}}}
{}
\pysigstopsignatures
\sphinxAtStartPar
\_summary\_
\begin{quote}\begin{description}
\sphinxlineitem{Paramètres}\begin{itemize}
\item {} 
\sphinxAtStartPar
\sphinxstyleliteralstrong{\sphinxupquote{model}} (\sphinxstyleliteralemphasis{\sphinxupquote{\_type\_}}) \textendash{} \_description\_

\item {} 
\sphinxAtStartPar
\sphinxstyleliteralstrong{\sphinxupquote{n}} (\sphinxstyleliteralemphasis{\sphinxupquote{int}}\sphinxstyleliteralemphasis{\sphinxupquote{, }}\sphinxstyleliteralemphasis{\sphinxupquote{optional}}) \textendash{} \_description\_, by default 10

\end{itemize}

\sphinxlineitem{Type renvoyé}
\sphinxAtStartPar
tuple contenant les moyennes et écart types de température, voltage et ph de la solution

\end{description}\end{quote}

\end{fulllineitems}

\index{measure() (dans le module lib\_pH)@\spxentry{measure()}\spxextra{dans le module lib\_pH}}

\begin{fulllineitems}
\phantomsection\label{\detokenize{library:lib_pH.measure}}
\pysigstartsignatures
\pysiglinewithargsret
{\sphinxcode{\sphinxupquote{lib\_pH.}}\sphinxbfcode{\sphinxupquote{measure}}}
{\sphinxparam{\DUrole{n}{model}}\sphinxparamcomma \sphinxparam{\DUrole{n}{n\_stab}\DUrole{o}{=}\DUrole{default_value}{20}}\sphinxparamcomma \sphinxparam{\DUrole{n}{port\_test}\DUrole{o}{=}\DUrole{default_value}{\textquotesingle{}\textquotesingle{}}}\sphinxparamcomma \sphinxparam{\DUrole{n}{n}\DUrole{o}{=}\DUrole{default_value}{10}}}
{}
\pysigstopsignatures
\sphinxAtStartPar
Mesure le pH en se basant sur une calibration et renvoie l’évolution des écart\sphinxhyphen{}type au cours du temps.

\sphinxAtStartPar
effectue n mesure individuelles
\begin{quote}\begin{description}
\sphinxlineitem{Paramètres}\begin{itemize}
\item {} 
\sphinxAtStartPar
\sphinxstyleliteralstrong{\sphinxupquote{model}} (\sphinxstyleliteralemphasis{\sphinxupquote{list}}\sphinxstyleliteralemphasis{\sphinxupquote{, }}\sphinxstyleliteralemphasis{\sphinxupquote{liste}}) \textendash{} Calibration utilisée. Par défaut les paramètres de courbe de calibration est a = 75.55116667 et b = \sphinxhyphen{}163.1275.

\item {} 
\sphinxAtStartPar
\sphinxstyleliteralstrong{\sphinxupquote{n}} (\sphinxstyleliteralemphasis{\sphinxupquote{int}}\sphinxstyleliteralemphasis{\sphinxupquote{, }}\sphinxstyleliteralemphasis{\sphinxupquote{nombre d\textquotesingle{}acquisitions pour une mesure}}) \textendash{} DESCRIPTION. The default is 10.

\item {} 
\sphinxAtStartPar
\sphinxstyleliteralstrong{\sphinxupquote{n\_stab}} (\sphinxstyleliteralemphasis{\sphinxupquote{int}}\sphinxstyleliteralemphasis{\sphinxupquote{, }}\sphinxstyleliteralemphasis{\sphinxupquote{nombre de mesures utilisées dans le calcul de stabilité}})

\item {} 
\sphinxAtStartPar
\sphinxstyleliteralstrong{\sphinxupquote{""}} (\sphinxstyleliteralemphasis{\sphinxupquote{port\_test =}})

\item {} 
\sphinxAtStartPar
\sphinxstyleliteralstrong{\sphinxupquote{string}}

\item {} 
\sphinxAtStartPar
\sphinxstyleliteralstrong{\sphinxupquote{ouvert}} (\sphinxstyleliteralemphasis{\sphinxupquote{port com}})

\end{itemize}

\end{description}\end{quote}

\end{fulllineitems}

\index{measurement() (dans le module lib\_pH)@\spxentry{measurement()}\spxextra{dans le module lib\_pH}}

\begin{fulllineitems}
\phantomsection\label{\detokenize{library:lib_pH.measurement}}
\pysigstartsignatures
\pysiglinewithargsret
{\sphinxcode{\sphinxupquote{lib\_pH.}}\sphinxbfcode{\sphinxupquote{measurement}}}
{\sphinxparam{\DUrole{n}{a}}\sphinxparamcomma \sphinxparam{\DUrole{n}{b}}\sphinxparamcomma \sphinxparam{\DUrole{n}{nb\_inter}}\sphinxparamcomma \sphinxparam{\DUrole{n}{time\_inter}}\sphinxparamcomma \sphinxparam{\DUrole{n}{s}}}
{}
\pysigstopsignatures
\sphinxAtStartPar
Mesure unique ou en série et enregistrement éventuel des données.
\begin{quote}\begin{description}
\sphinxlineitem{Paramètres}\begin{itemize}
\item {} 
\sphinxAtStartPar
\sphinxstyleliteralstrong{\sphinxupquote{a}} (\sphinxstyleliteralemphasis{\sphinxupquote{float}}) \textendash{} Pente de régression linéaire entre pH et voltage mesuré.

\item {} 
\sphinxAtStartPar
\sphinxstyleliteralstrong{\sphinxupquote{b}} (\sphinxstyleliteralemphasis{\sphinxupquote{float}}) \textendash{} Ordonnée à l’origine de régression linéaire entre pH et voltage mesuré.

\item {} 
\sphinxAtStartPar
\sphinxstyleliteralstrong{\sphinxupquote{nb\_inter}} (\sphinxstyleliteralemphasis{\sphinxupquote{int}}) \textendash{} Nombre de valeurs utilisées pour constituer une mesure (une mesure correspond à la moyenne de toutes les valeurs prélevées).

\item {} 
\sphinxAtStartPar
\sphinxstyleliteralstrong{\sphinxupquote{time\_inter}} (\sphinxstyleliteralemphasis{\sphinxupquote{float}}) \textendash{} Temps d’intervalle entre chaque prélèvement de valeur au sein d’une mesure.

\item {} 
\sphinxAtStartPar
\sphinxstyleliteralstrong{\sphinxupquote{s}} (\sphinxstyleliteralemphasis{\sphinxupquote{serial.tools.list\_ports\_common.ListPortInfo}}) \textendash{} Objet Serial sur lequel on peut appliquer des fonctions d’ouverture, de lecture et de fermeture du port série affilié.

\end{itemize}

\sphinxlineitem{Type renvoyé}
\sphinxAtStartPar
None.

\end{description}\end{quote}

\end{fulllineitems}

\index{pH\_sensor() (dans le module lib\_pH)@\spxentry{pH\_sensor()}\spxextra{dans le module lib\_pH}}

\begin{fulllineitems}
\phantomsection\label{\detokenize{library:lib_pH.pH_sensor}}
\pysigstartsignatures
\pysiglinewithargsret
{\sphinxcode{\sphinxupquote{lib\_pH.}}\sphinxbfcode{\sphinxupquote{pH\_sensor}}}
{\sphinxparam{\DUrole{n}{nb\_inter}}\sphinxparamcomma \sphinxparam{\DUrole{n}{time\_inter}}\sphinxparamcomma \sphinxparam{\DUrole{n}{s}}}
{}
\pysigstopsignatures
\sphinxAtStartPar
Mesure du voltage du pH\sphinxhyphen{}mètre et de la température.
\begin{quote}\begin{description}
\sphinxlineitem{Paramètres}\begin{itemize}
\item {} 
\sphinxAtStartPar
\sphinxstyleliteralstrong{\sphinxupquote{nb\_inter}} (\sphinxstyleliteralemphasis{\sphinxupquote{int}}) \textendash{} Nombre de valeurs utilisées pour constituer une mesure (une mesure correspond à la moyenne de toutes les valeurs prélevées).

\item {} 
\sphinxAtStartPar
\sphinxstyleliteralstrong{\sphinxupquote{time\_inter}} (\sphinxstyleliteralemphasis{\sphinxupquote{float}}) \textendash{} Temps d’intervalle entre chaque prélèvement de valeur au sein d’une mesure.

\item {} 
\sphinxAtStartPar
\sphinxstyleliteralstrong{\sphinxupquote{s}} (\sphinxstyleliteralemphasis{\sphinxupquote{serial.tools.list\_ports\_common.ListPortInfo}}) \textendash{} Objet Serial sur lequel on peut appliquer des fonctions d’ouverture, de lecture et de fermeture du port série affilié.

\end{itemize}

\sphinxlineitem{Renvoie}
\sphinxAtStartPar
\begin{itemize}
\item {} 
\sphinxAtStartPar
\sphinxstylestrong{list\_pH} (\sphinxstyleemphasis{list}) \textendash{} valeurs de voltages liées au pH et utilisées pour une mesure.

\item {} 
\sphinxAtStartPar
\sphinxstylestrong{list\_temperatures} (\sphinxstyleemphasis{list}) \textendash{} valeurs de températures utilisées pour une mesure.

\end{itemize}


\end{description}\end{quote}

\end{fulllineitems}

\index{pH\_temp\_adjust() (dans le module lib\_pH)@\spxentry{pH\_temp\_adjust()}\spxextra{dans le module lib\_pH}}

\begin{fulllineitems}
\phantomsection\label{\detokenize{library:lib_pH.pH_temp_adjust}}
\pysigstartsignatures
\pysiglinewithargsret
{\sphinxcode{\sphinxupquote{lib\_pH.}}\sphinxbfcode{\sphinxupquote{pH\_temp\_adjust}}}
{\sphinxparam{\DUrole{n}{pH}}\sphinxparamcomma \sphinxparam{\DUrole{n}{temp}}}
{}
\pysigstopsignatures
\sphinxAtStartPar
Ajustement du pH étalon en fonction de la température par interpolation linéaire.
\begin{quote}\begin{description}
\sphinxlineitem{Paramètres}\begin{itemize}
\item {} 
\sphinxAtStartPar
\sphinxstyleliteralstrong{\sphinxupquote{pH}} (\sphinxstyleliteralemphasis{\sphinxupquote{int}}) \textendash{} pH de la solution étalon.

\item {} 
\sphinxAtStartPar
\sphinxstyleliteralstrong{\sphinxupquote{temp}} (\sphinxstyleliteralemphasis{\sphinxupquote{float}}) \textendash{} température mesurée de la solution.

\end{itemize}

\sphinxlineitem{Renvoie}
\sphinxAtStartPar
\sphinxstylestrong{pH\_adjusted} \textendash{} pH interpolé en fonction de correspondances entre pH et températures connues.

\sphinxlineitem{Type renvoyé}
\sphinxAtStartPar
float

\end{description}\end{quote}

\end{fulllineitems}

\index{plot\_calib() (dans le module lib\_pH)@\spxentry{plot\_calib()}\spxextra{dans le module lib\_pH}}

\begin{fulllineitems}
\phantomsection\label{\detokenize{library:lib_pH.plot_calib}}
\pysigstartsignatures
\pysiglinewithargsret
{\sphinxcode{\sphinxupquote{lib\_pH.}}\sphinxbfcode{\sphinxupquote{plot\_calib}}}
{\sphinxparam{\DUrole{n}{voltage\_values}}\sphinxparamcomma \sphinxparam{\DUrole{n}{buffers}}\sphinxparamcomma \sphinxparam{\DUrole{n}{errorbuffers\_values}}\sphinxparamcomma \sphinxparam{\DUrole{n}{errorvoltage\_values}}\sphinxparamcomma \sphinxparam{\DUrole{n}{t}}\sphinxparamcomma \sphinxparam{\DUrole{n}{EM4}}\sphinxparamcomma \sphinxparam{\DUrole{n}{EM7}}\sphinxparamcomma \sphinxparam{\DUrole{n}{EM10}}\sphinxparamcomma \sphinxparam{\DUrole{n}{predict}}\sphinxparamcomma \sphinxparam{\DUrole{n}{equation}}\sphinxparamcomma \sphinxparam{\DUrole{n}{R2}}}
{}
\pysigstopsignatures
\sphinxAtStartPar
Représentation graphique des calibrations
\begin{quote}\begin{description}
\sphinxlineitem{Paramètres}\begin{itemize}
\item {} 
\sphinxAtStartPar
\sphinxstyleliteralstrong{\sphinxupquote{voltage\_values}} (\sphinxstyleliteralemphasis{\sphinxupquote{\_type\_}}) \textendash{} valeurs de voltages de l’arduino

\item {} 
\sphinxAtStartPar
\sphinxstyleliteralstrong{\sphinxupquote{buffers}} (\sphinxstyleliteralemphasis{\sphinxupquote{\_type\_}}) \textendash{} valeurs des tampons

\item {} 
\sphinxAtStartPar
\sphinxstyleliteralstrong{\sphinxupquote{errorbuffers\_values}} (\sphinxstyleliteralemphasis{\sphinxupquote{\_type\_}}) \textendash{} incertitudes sur les tampons

\item {} 
\sphinxAtStartPar
\sphinxstyleliteralstrong{\sphinxupquote{errorvoltage\_values}} (\sphinxstyleliteralemphasis{\sphinxupquote{\_type\_}}) \textendash{} incertitude sur les voltages

\item {} 
\sphinxAtStartPar
\sphinxstyleliteralstrong{\sphinxupquote{t}} (\sphinxstyleliteralemphasis{\sphinxupquote{\_type\_}}) \textendash{} temps

\item {} 
\sphinxAtStartPar
\sphinxstyleliteralstrong{\sphinxupquote{EM4}} (\sphinxstyleliteralemphasis{\sphinxupquote{\_type\_}}) \textendash{} \_description\_

\item {} 
\sphinxAtStartPar
\sphinxstyleliteralstrong{\sphinxupquote{EM7}} (\sphinxstyleliteralemphasis{\sphinxupquote{\_type\_}}) \textendash{} \_description\_

\item {} 
\sphinxAtStartPar
\sphinxstyleliteralstrong{\sphinxupquote{EM10}} (\sphinxstyleliteralemphasis{\sphinxupquote{\_type\_}}) \textendash{} \_description\_

\item {} 
\sphinxAtStartPar
\sphinxstyleliteralstrong{\sphinxupquote{predict}} (\sphinxstyleliteralemphasis{\sphinxupquote{\_type\_}}) \textendash{} fonction de prédiction pH=f(V)

\item {} 
\sphinxAtStartPar
\sphinxstyleliteralstrong{\sphinxupquote{equation}} (\sphinxstyleliteralemphasis{\sphinxupquote{\_type\_}}) \textendash{} equation de la calibration

\item {} 
\sphinxAtStartPar
\sphinxstyleliteralstrong{\sphinxupquote{R2}} (\sphinxstyleliteralemphasis{\sphinxupquote{\_type\_}}) \textendash{} R2 de la calibration

\end{itemize}

\end{description}\end{quote}

\end{fulllineitems}

\index{plot\_mes() (dans le module lib\_pH)@\spxentry{plot\_mes()}\spxextra{dans le module lib\_pH}}

\begin{fulllineitems}
\phantomsection\label{\detokenize{library:lib_pH.plot_mes}}
\pysigstartsignatures
\pysiglinewithargsret
{\sphinxcode{\sphinxupquote{lib\_pH.}}\sphinxbfcode{\sphinxupquote{plot\_mes}}}
{\sphinxparam{\DUrole{n}{T}}}
{}
\pysigstopsignatures
\sphinxAtStartPar
Représentation graphique des séries de mesures
\begin{quote}\begin{description}
\sphinxlineitem{Paramètres}
\sphinxAtStartPar
\sphinxstyleliteralstrong{\sphinxupquote{T}} (\sphinxstyleliteralemphasis{\sphinxupquote{str}}) \textendash{} date du fichier

\end{description}\end{quote}

\end{fulllineitems}

\index{port\_connexion() (dans le module lib\_pH)@\spxentry{port\_connexion()}\spxextra{dans le module lib\_pH}}

\begin{fulllineitems}
\phantomsection\label{\detokenize{library:lib_pH.port_connexion}}
\pysigstartsignatures
\pysiglinewithargsret
{\sphinxcode{\sphinxupquote{lib\_pH.}}\sphinxbfcode{\sphinxupquote{port\_connexion}}}
{\sphinxparam{\DUrole{n}{br}\DUrole{o}{=}\DUrole{default_value}{9600}}\sphinxparamcomma \sphinxparam{\DUrole{n}{portIN}\DUrole{o}{=}\DUrole{default_value}{\textquotesingle{}\textquotesingle{}}}}
{}
\pysigstopsignatures
\sphinxAtStartPar
Établit la connexion au port série.
\begin{quote}\begin{description}
\sphinxlineitem{Paramètres}\begin{itemize}
\item {} 
\sphinxAtStartPar
\sphinxstyleliteralstrong{\sphinxupquote{br}} (\sphinxstyleliteralemphasis{\sphinxupquote{int}}) \textendash{} Flux de données en baud.

\item {} 
\sphinxAtStartPar
\sphinxstyleliteralstrong{\sphinxupquote{portIN}} (\sphinxstyleliteralemphasis{\sphinxupquote{string}}) \textendash{} Identifiant du port série sur lequel le script doit lire des données.

\end{itemize}

\sphinxlineitem{Renvoie}
\sphinxAtStartPar
\begin{itemize}
\item {} 
\sphinxAtStartPar
\sphinxstylestrong{port} (\sphinxstyleemphasis{string}) \textendash{} Identifiant du port série sur lequel le script doit lire des données.

\item {} 
\sphinxAtStartPar
\sphinxstylestrong{s} (\sphinxstyleemphasis{serial.tools.list\_ports\_common.ListPortInfo / string}) \textendash{} Objet Serial sur lequel on peut appliquer des fonctions d’ouverture, de lecture et de fermeture du port série affilié. En cas d’échec de connexion, “s” sera une chaîne de caractères « erreur ».

\end{itemize}


\end{description}\end{quote}

\end{fulllineitems}

\begin{itemize}
\item {} 
\sphinxAtStartPar
\DUrole{xref}{\DUrole{std}{\DUrole{std-ref}{genindex}}}

\item {} 
\sphinxAtStartPar
\DUrole{xref}{\DUrole{std}{\DUrole{std-ref}{modindex}}}

\item {} 
\sphinxAtStartPar
\DUrole{xref}{\DUrole{std}{\DUrole{std-ref}{search}}}

\end{itemize}


\renewcommand{\indexname}{Index des modules Python}
\begin{sphinxtheindex}
\let\bigletter\sphinxstyleindexlettergroup
\bigletter{l}
\item\relax\sphinxstyleindexentry{lib\_pH}\sphinxstyleindexpageref{library:\detokenize{module-lib_pH}}
\end{sphinxtheindex}

\renewcommand{\indexname}{Index}
\printindex
\end{document}